\documentclass[a4paper, 12pt]{article}
\usepackage[utf8]{inputenc}
\usepackage[T1]{fontenc}
\usepackage{amsmath}
\usepackage{amssymb}
\usepackage{listings}
\usepackage{xcolor}
\usepackage{geometry}
\usepackage{fancyhdr}
\usepackage{hyperref}
\usepackage{verbatim}
\usepackage{graphicx}
\usepackage{longtable}
\usepackage{todonotes}

\pagestyle{fancy}

\geometry{margin=1in}

% Java code setup
\lstset{
    language=Java, % Language specification
    basicstyle=\ttfamily\small, % Code font size and type
    keywordstyle=\color{blue}\bfseries, % Keywords color
    stringstyle=\color{teal}, % Strings color
    commentstyle=\color{olive}, % Comments color
    numbers=left, % Line numbers
    numberstyle=\tiny\color{gray}, % Line numbers style
    stepnumber=1, % The step between two line numbers
    numbersep=5pt, % How far the line numbers are from the code
    backgroundcolor=\color{lightgray!10}, % Background color for the code block
    showspaces=false, % Show spaces in the code
    showstringspaces=false, % Don't show spaces in strings
    tabsize=4, % Tab size
    breaklines=true, % Line breaking
    breakatwhitespace=true, % Break lines only at whitespace
    frame=single, % Adds a frame around the code
    captionpos=b, % Caption position: 't' for top and 'b' for bottom
}
%\begin{lstlisting}[language=yaml]
\lstdefinelanguage{yaml}{
    keywords={true, false, null, yes, no},
    keywordstyle=\color{blue}\bfseries,
    basicstyle=\ttfamily,
    sensitive=false,
    comment=[l]{\#},
    commentstyle=\color{gray}\ttfamily,
    morestring=[b]',
    morestring=[b]",
    stringstyle=\color{orange},
    literate =    {---}{{\processthree}}3
        {>}{{\textcolor{red}{>}}}1
        {|}{{\textcolor{red}{|}}}1
        {*}{{\textcolor{red}{*}}}1
        {:}{{\textcolor{red}{:}}}1,
}

% Header configuration
\fancyhead[L]{\href{https://github.com/Jok98}{GitHub}}
\fancyhead[R]{\href{https://www.linkedin.com/in/matteo-moi/}{LinkedIn}}

\begin{document}

    \title{DevOps}
    \author{Matteo Moi}
    \date{}
    \maketitle

    \tableofcontents
    \newpage


    \section{Docker}
    \textit{\textbf{Definition}}: is a platform that allows developers to package applications into containers—standardized units of software that include everything the application needs to run.
    A container is actually a process in a Linux host that use Linux namespaces to provide isolation between different containers.

    \subsection{How to set up an environment with Docker}

    \subsubsection{Create a Dockerfile}
    \begin{lstlisting}
# Use a base image with Java
FROM openjdk:17-jdk-alpine

# Set the working directory inside the container
WORKDIR /app

# Copy the compiled JAR file from the target directory to the container
COPY target/myapp-0.0.1-SNAPSHOT.jar /app/myapp.jar

# Expose the port that the Spring Boot application will run on
EXPOSE 8080

# Define the command to run the application
ENTRYPOINT ["java", "-jar", "/app/myapp.jar"]

    \end{lstlisting}

    Build the Docker Image
    \begin{lstlisting}
docker build -t myapp:latest .
    \end{lstlisting}

    Run the Docker Container
    \begin{lstlisting}
docker run -p 8080:8080 myapp:latest
    \end{lstlisting}
    http://localhost:8080

    \subsubsection{Docker Compose for Multi-Container Setup}
    Docker Compose allows you to define and run multi-container Docker applications using a docker-compose.yml file.

    \begin{lstlisting}[language=yaml]
version: '3'

services:
  service1:
    image: service1:latest
    build:
      context: ./service1
    ports:
      - "8081:8080"
    networks:
      - microservices-network

  service2:
    image: service2:latest
    build:
      context: ./service2
    ports:
      - "8082:8080"
    networks:
      - microservices-network

networks:
  microservices-network:
    driver: bridge

    \end{lstlisting}
    \begin{itemize}
        \item \textbf{services}: Defines the microservices and their configurations.
        \item \textbf{image}: Specifies the image for each service.
        \item \textbf{build}: Defines the build context for each service (the directory containing the Dockerfile).
        \item \textbf{ports}: Maps the container ports to the host machine.
        \item \textbf{networks}: Creates an isolated network for the microservices to communicate with each other.
    \end{itemize}
    To start all the services
    \begin{lstlisting}
docker-compose up --build
    \end{lstlisting}


    \section{in progress...}
\end{document}