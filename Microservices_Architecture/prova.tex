\documentclass{article}
\usepackage[utf8]{inputenc} % Permette l'uso dei caratteri UTF-8
\usepackage{listings} % Pacchetto per inserire codice
\usepackage{xcolor} % Pacchetto per colori

% Configurazione del pacchetto listings per Java
\lstset{
    language=Java,                % Linguaggio del codice
    basicstyle=\ttfamily\small,   % Stile del testo
    keywordstyle=\color{blue},    % Colore delle parole chiave
    stringstyle=\color{red},      % Colore delle stringhe
    commentstyle=\color{green},   % Colore dei commenti
    numbers=left,                 % Numeri di riga a sinistra
    numberstyle=\tiny\color{gray},% Stile dei numeri di riga
    stepnumber=1,                 % Ogni riga numerata
    tabsize=2,                    % Dimensione del tab
    showspaces=false,             % Non mostra gli spazi
    showstringspaces=false,       % Non mostra gli spazi nelle stringhe
}

\title{Inserire Codice Java in LaTeX}
\author{Il Tuo Nome}
\date{\today}

\begin{document}

    \maketitle

    \section{Esempio di Codice Java}

    Di seguito è riportato un esempio di codice Java con i metodi in viola:

    \begin{lstlisting}
// Questo è un semplice programma Java
public class HelloWorld {
    public static void main(String[] args) {
        System.out.println("Hello, World!");
    }

    private void metodoPrivato() {
        // Metodo privato
    }
}
    \end{lstlisting}

\end{document}
