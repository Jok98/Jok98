\documentclass[a4paper, 12pt]{article}
\usepackage[utf8]{inputenc}
\usepackage{amsmath}
\usepackage{amssymb}
\usepackage{listings}
\usepackage{xcolor}
\usepackage{geometry}
\usepackage{fancyhdr}
\usepackage{hyperref}

\pagestyle{fancy}

\geometry{margin=1in}

% Java code setup
\lstset{
    language=Java, % Language specification
    basicstyle=\ttfamily\small, % Code font size and type
    keywordstyle=\color{blue}\bfseries, % Keywords color
    stringstyle=\color{teal}, % Strings color
    commentstyle=\color{olive}, % Comments color
    numbers=left, % Line numbers
    numberstyle=\tiny\color{gray}, % Line numbers style
    stepnumber=1, % The step between two line numbers
    numbersep=5pt, % How far the line numbers are from the code
    backgroundcolor=\color{lightgray!10}, % Background color for the code block
    showspaces=false, % Show spaces in the code
    showstringspaces=false, % Don't show spaces in strings
    tabsize=4, % Tab size
    breaklines=true, % Line breaking
    breakatwhitespace=true, % Break lines only at whitespace
    frame=single, % Adds a frame around the code
    captionpos=b, % Caption position: 't' for top and 'b' for bottom
}

\lstdefinelanguage{yaml}{
    keywords={true, false, null, yes, no},
    keywordstyle=\color{blue}\bfseries,
    basicstyle=\ttfamily,
    sensitive=false,
    comment=[l]{\#},
    commentstyle=\color{gray}\ttfamily,
    morestring=[b]',
    morestring=[b]",
    stringstyle=\color{orange},
    literate =    {---}{{\processthree}}3
        {>}{{\textcolor{red}{>}}}1
        {|}{{\textcolor{red}{|}}}1
        {*}{{\textcolor{red}{*}}}1
        {:}{{\textcolor{red}{:}}}1,
}


% GitHub and LinkedIn URLs
\newcommand{\githuburl}{https://github.com/yourusername}
\newcommand{\linkedinurl}{https://linkedin.com/in/yourusername}

% Header configuration
\fancyhead[L]{\href{https://github.com/Jok98}{GitHub}}
\fancyhead[R]{\href{https://www.linkedin.com/in/matteo-moi/}{LinkedIn}}

\begin{document}

    \title{Api Gateway}
    \author{Matteo Moi}
    \date{}
    \maketitle

    \tableofcontents
    \newpage

% Start your notes here


    \section{Introduction}

    An API Gateway is a critical component in a microservices architecture, acting as a single entry point for client requests and routing them to the appropriate backend services.

    \subsection{Features}
    \begin{itemize}
        \item \textbf{Routing: }It routes incoming requests to the appropriate microservice. It abstracts the complexity of the underlying microservices, allowing clients to interact with a unified API
        \item \textbf{Load Balancing: }It distributes incoming requests across multiple instances of a service to ensure optimal resource utilization and high availability, improving performance and reliability.
        \item \textbf{Protocol Translation: }It can translate protocols, enabling clients to use a simple protocol like HTTP/HTTPS while the backend services may use other protocols such as gRPC, WebSocket, or SOAP. This allows seamless communication between different types of services and clients.
        \item \textbf{Authentication and Authorization: }It provides a centralized point for implementing authentication and authorization. It can integrate with OAuth, JWT, API keys.
        \item \textbf{Rate Limiting: }To protect services from being overwhelmed, the API Gateway can enforce rate limits and throttling policies, controlling the number of requests a client can make within a specified period.
        \item \textbf{Transformation: } It can transform incoming requests before forwarding them to the backend services. This includes modifying headers, rewriting URLs, or transforming the request payload.
        \item \textbf{Aggregation: }It can aggregate responses from multiple microservices into a single response, reducing the number of round trips between the client and server.
        \item \textbf{Logging: }The API Gateway can collect logs, metrics, and other analytics data. This includes tracking request rates, response times, error rates, and more, providing insights into the system's performance and health.
        \item \textbf{Caching: }It can cache responses from backend services to improve performance and reduce the load on services.
        \item \textbf{Cross-Origin Resource Sharing (CORS): }It can manage CORS policies, enabling or restricting access to resources from different origins.
        \item \textbf{Failover: }It can handle failover scenarios by routing requests to backup services or returning cached responses if the primary service is unavailable.
        \item \textbf{Circuit Breaking: }It can implement circuit breaker patterns to prevent cascading failures in the system by temporarily blocking requests to failing services and allowing them to recover.
        \item \textbf{Service Discovery Integration: }It can integrate with service discovery mechanisms to dynamically route requests to the correct instances of microservices. This ensures that the gateway always knows the available services and their locations.
        \item \textbf{API Versioning: }It can handle different versions of APIs, allowing clients to use specific versions while enabling the development of new versions without breaking existing clients.
    \end{itemize}

    \subsection*{Example of API Gateways}
    \begin{itemize}
        \item \textbf{NGINX}
        \item \textbf{AWS API Gateway}
        \item \textbf{Netflix Zuul}
        \item \textbf{Spring Cloud Gateway}
        \item \textbf{Kong}
    \end{itemize}


\end{document}
